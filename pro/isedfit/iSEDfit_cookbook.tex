\documentclass[12pt,preprint]{aastex}
\usepackage{fancyhdr, mdwlist}

\newcommand{\ised}{{\tt iSEDfit}}
\newcommand{\sfhpro}{{\tt write\_sfhgrid\_paramfile}}
\newcommand{\superpro}{{\tt write\_supergrid\_paramfile}}

\pagestyle{fancy}
\lhead{\ised{} Cookbook}
\rhead{Version 0.1}

\begin{document}

\vspace*{-10mm} 
\begin{center}
{\Large {\bf \ised}} \\
\vspace*{2mm} 
{\Large {\bf\sc Version 0.1}} \\
\vspace*{2mm} 
{\Large {\bf\sc Cookbook}} \\
\vspace*{0.1in} 
{\Large {\bf\sc John Moustakas}} \\
{\Large {\bf\sc Siena College}} \\
\vspace*{0.1in} 
\end{center}

\section{What is \ised?}

Table of contents, intro stuff.

\section{License}

This is the license.

\section{ToDo}

\begin{enumerate}
\item{Version the SSPs}
\end{enumerate}

\section{Installation}

Define the requisite environment variables:

\begin{verbatim}
setenv ISEDFIT_SSP_DIR ${IM_DATA_DIR}/isedfit_ssp
\end{verbatim}


Download the prebuilt SSPs:

\begin{verbatim}
http://cass.ucsd.edu/~ioannis/isedfit/isedfit_ssp.tar.gz
\end{verbatim}

Make the directories
montegrids
isedfit

Step-by-step instructions:

Intro: iSEDfit requires three different parameter files.  A global
parameter file that is tied to a particular dataset (e.g., it has
filter information); a "sfhgrid" (star formation history grid)
parameter file, which specifies the prior parameter choices on the
star formation history, metallicity, attenuation, burst parameters,
etc.; and a "supergrid" parameter file, which specifies the SPS
models, IMF, and attenuation/reddening curve.

This data model was chosen because it allows for significant
flexibility in exploring different combinations of priors on the
results.  In the first section I will describe the simplest possible
setup to get the code up and running, and in subsequent sections I
will give many other examples of how to run iSEDfit.

\section{Getting Up and Running}

1) First make the global parameter file.  Choose the filter list, and
be sure they have been incorporated into K-correct.  Choose the
minimum and maximum redshifts you will consider.  Choose a unique
prefix for the project.

Then run: 

\begin{verbatim}
write_isedfit_paramfile, filterlist, prefix=prefix, minz=minz, maxz=maxz, $
  nzz=nzz, zlog=zlog, h100=h100, omega0=omega0, omegal=omegal, igm=igm, $
  isedpath=isedpath, clobber=clobber
\end{verbatim}

This will write out a parameter file in some directory....

2) Next build the "sfhgrid" parameter file.  This file specifies your
choice of star formation history (SFH) priors.  You can have as many
combinations of priors as you want and they will each be assigned a
unique identifier.  The routine that will need to be called is

\begin{verbatim}
write_sfhgrid_paramfile, 
\end{verbatim}


This routine will need to be called once per SFHGRID (combination of
priors).  An existing parameter file can either be overwritten or
appended to.
  

Have the code output QAplots of the prior parameter distributions, or
write a code that will do that.

Note that for some models the stellar metallicity will be truncated
outside the range where the models have been calibrated.

Describe the gamma distributions used for $A_V$ and $\mu$; show some
plots with the parameters varied.

The code determines if bursts are desired if PBURST is greater than
zero.  So even if the other burst parameters have values (sensible
defaults), they are ignored if pburst is less than or equal to zero.
And obviously pburst should be less than or equal to unity.
  

3) Next build the "supergrid" parameter file.  This file relates each
SFHGRID to a particular choice of SPS models, IMF, and attenuation
curve.  The strength of this data model is that you can easily explore
the effect of different SPS models or attenuation curve on your
results.

\begin{verbatim}
write_supergrid_paramfile, 
\end{verbatim}


%   # Parameter set describing the "supergrids" to consider
%   
%   typedef struct {
%    int supergrid;        # supergrid number
%    int sfhgrid;          # SFHgrid number (see clash_sfhgrid.par)
%    char imf;             # IMF
%    char synthmodels;     # choice of population synthesis models
%    int redcurve;         # reddening curve (0=calzetti; 1=charlot)
%   } GRID;
%   
%   GRID   1   1 chab     fsps      0


Doing More Complication Things

1) Exploring more than one SFHGRID.  Need to call \sfhpro{} more than
once.  Example: bursts and no bursts.

2) Exploring two different SPS models and two attenuation curves while
keeping the SFHGRID priors fixed.  Call \superpro{} again.

3) Changing your redshift range while keeping everything else
constant.  Example: exploring a low-redshift solution.


ToDo



\end{document}
